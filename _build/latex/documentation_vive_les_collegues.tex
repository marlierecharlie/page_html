%% Generated by Sphinx.
\def\sphinxdocclass{report}
\documentclass[letterpaper,10pt,french]{sphinxmanual}
\ifdefined\pdfpxdimen
   \let\sphinxpxdimen\pdfpxdimen\else\newdimen\sphinxpxdimen
\fi \sphinxpxdimen=.75bp\relax
\ifdefined\pdfimageresolution
    \pdfimageresolution= \numexpr \dimexpr1in\relax/\sphinxpxdimen\relax
\fi
%% let collapsible pdf bookmarks panel have high depth per default
\PassOptionsToPackage{bookmarksdepth=5}{hyperref}

\PassOptionsToPackage{booktabs}{sphinx}
\PassOptionsToPackage{colorrows}{sphinx}

\PassOptionsToPackage{warn}{textcomp}
\usepackage[utf8]{inputenc}
\ifdefined\DeclareUnicodeCharacter
% support both utf8 and utf8x syntaxes
  \ifdefined\DeclareUnicodeCharacterAsOptional
    \def\sphinxDUC#1{\DeclareUnicodeCharacter{"#1}}
  \else
    \let\sphinxDUC\DeclareUnicodeCharacter
  \fi
  \sphinxDUC{00A0}{\nobreakspace}
  \sphinxDUC{2500}{\sphinxunichar{2500}}
  \sphinxDUC{2502}{\sphinxunichar{2502}}
  \sphinxDUC{2514}{\sphinxunichar{2514}}
  \sphinxDUC{251C}{\sphinxunichar{251C}}
  \sphinxDUC{2572}{\textbackslash}
\fi
\usepackage{cmap}
\usepackage[T1]{fontenc}
\usepackage{amsmath,amssymb,amstext}
\usepackage{babel}



\usepackage{tgtermes}
\usepackage{tgheros}
\renewcommand{\ttdefault}{txtt}



\usepackage[Sonny]{fncychap}
\ChNameVar{\Large\normalfont\sffamily}
\ChTitleVar{\Large\normalfont\sffamily}
\usepackage{sphinx}

\fvset{fontsize=auto}
\usepackage{geometry}


% Include hyperref last.
\usepackage{hyperref}
% Fix anchor placement for figures with captions.
\usepackage{hypcap}% it must be loaded after hyperref.
% Set up styles of URL: it should be placed after hyperref.
\urlstyle{same}

\addto\captionsfrench{\renewcommand{\contentsname}{Contenu:}}

\usepackage{sphinxmessages}
\setcounter{tocdepth}{1}



\title{Documentation\_Vive\_les\_collegues}
\date{janv. 08, 2023}
\release{05/01/2023}
\author{Charlie}
\newcommand{\sphinxlogo}{\vbox{}}
\renewcommand{\releasename}{Version}
\makeindex
\begin{document}

\ifdefined\shorthandoff
  \ifnum\catcode`\=\string=\active\shorthandoff{=}\fi
  \ifnum\catcode`\"=\active\shorthandoff{"}\fi
\fi

\pagestyle{empty}
\sphinxmaketitle
\pagestyle{plain}
\sphinxtableofcontents
\pagestyle{normal}
\phantomsection\label{\detokenize{index::doc}}
\sphinxstepscope



\sphinxAtStartPar
1\_Allez sur la page \sphinxurl{https://github.com/marlierecharlie/brief\_15\_j\_aime\_mes\_commentaires.git}

\sphinxAtStartPar
2\_cliquez sur « clone » et copiez le lien https

\sphinxAtStartPar
3\_Dans git bash, tapez « cd », puis copiez le chemin du dossier dans lequel vous souhaitez l’installer.

\sphinxAtStartPar
4\_Toujours dans gitbash, tapez « git clone » + »le lien https cloné ».

\sphinxAtStartPar
Vous pouvez ensuite aller télécharger les packages nécessaires.

\sphinxstepscope


\chapter{Liste des packages a installer}
\label{\detokenize{Liste_des_packages_a_installer:liste-des-packages-a-installer}}\label{\detokenize{Liste_des_packages_a_installer::doc}}
\sphinxAtStartPar
Afin de pouvoir faire fonctionner le logiciel, il vous faut installer certains packages.

\sphinxAtStartPar
Pour ce faire, dans votre fenetre de terminal, copiez:

\begin{sphinxVerbatim}[commandchars=\\\{\}]
\PYG{n}{pip} \PYG{n}{install} \PYG{n}{pandas}
\PYG{n}{pip} \PYG{n}{install} \PYG{n}{numpy}
\PYG{n}{pip} \PYG{n}{install} \PYG{n}{inquirer}
\end{sphinxVerbatim}

\sphinxAtStartPar
Vous pouvez ensuite aller consulter la procédure d’utilisation du logiciel.

\sphinxstepscope


\chapter{Procédure d’utilisation}
\label{\detokenize{Procedure_utilisation:procedure-d-utilisation}}\label{\detokenize{Procedure_utilisation::doc}}\begin{description}
\sphinxlineitem{::}
\sphinxAtStartPar
1\sphinxhyphen{} remplir le champ du nom

\sphinxAtStartPar
2\sphinxhyphen{} remplir le champ du prénom

\sphinxAtStartPar
3\sphinxhyphen{} indiquer la date d’embauche

\sphinxAtStartPar
4\sphinxhyphen{} indiquer le poste du nouvel employé

\sphinxAtStartPar
5\sphinxhyphen{} indiquer votre nom et votre prénom Votre ajout s’implémentera à la fin de la liste des employés

\sphinxAtStartPar
6\sphinxhyphen{} faites un mail au service rh \sphinxhref{mailto:'rh.mail@gmail.com}{“rh.mail@gmail.com}” avec pour objet “ajout nouvel employé”

\sphinxAtStartPar
7\sphinxhyphen{} faites un mail au gestionnaire informatique \sphinxhref{mailto:'charlie.marliere@gmail.com}{“charlie.marliere@gmail.com}” avec pour objet “ajout nouvel employés” afin de faciliter la gestion et corriger les doublons

\sphinxAtStartPar
8\sphinxhyphen{} Bravo, vous avez ajouté un employé sur votre ordinateur, localement, mais vous devez renvoyer les informations sur git afin que toute l’entreprise puisse y accéder.
Conultez donc la rubrique suivante « Faire un merge request ».

\end{description}

\sphinxAtStartPar
En cas de doute, de question ou de besoin écrivez un mail au service rh (\sphinxhref{mailto:rh.mail@gmail.com}{rh.mail@gmail.com}) pour les questions administratives et au gestionnaire informatique (\sphinxhref{mailto:charlie.marliere@gmail.com}{charlie.marliere@gmail.com}) pour les questions sur le remplissage du fichier. Mettez l’autre service en copie.

\sphinxstepscope


\chapter{Faire un merge request}
\label{\detokenize{faire_un_merge_request:faire-un-merge-request}}\label{\detokenize{faire_un_merge_request::doc}}
\sphinxAtStartPar
Pour faire un merge request, vous avez 2 possibilites :

\sphinxAtStartPar
Avec un ticket (issue)

\begin{sphinxVerbatim}[commandchars=\\\{\}]
\PYG{l+m+mi}{1}\PYG{p}{)} \PYG{n}{Creer} \PYG{n}{un} \PYG{n}{ticket} \PYG{p}{(}\PYG{n}{issue}\PYG{p}{)}
\PYG{l+m+mi}{2}\PYG{p}{)} \PYG{n}{Clique} \PYG{n}{sur} \PYG{n}{la} \PYG{n}{petite} \PYG{n}{fleche} \PYG{n}{de} \PYG{l+s+s1}{\PYGZsq{}}\PYG{l+s+s1}{Create merge request}\PYG{l+s+s1}{\PYGZsq{}} \PYG{p}{(}\PYG{n}{bouton} \PYG{n}{bleu}\PYG{p}{)}
\PYG{l+m+mi}{3}\PYG{p}{)} \PYG{n}{Renommer} \PYG{n}{la} \PYG{n}{partie} \PYG{l+s+s1}{\PYGZsq{}}\PYG{l+s+s1}{Branch name}\PYG{l+s+s1}{\PYGZsq{}} \PYG{n}{par} \PYG{n}{celui} \PYG{n}{que} \PYG{n}{vous} \PYG{n}{voulez}
\PYG{l+m+mi}{4}\PYG{p}{)} \PYG{n}{Validez} \PYG{n}{en} \PYG{n}{cliquant} \PYG{n}{sur} \PYG{l+s+s1}{\PYGZsq{}}\PYG{l+s+s1}{Create merge request}\PYG{l+s+s1}{\PYGZsq{}}

\PYG{o}{\PYGZhy{}}\PYG{o}{\PYGZhy{}}\PYG{o}{\PYGZhy{}}\PYG{o}{\PYGZhy{}} \PYG{n}{Passez} \PYG{n}{sur} \PYG{n}{GIT} \PYG{n}{Bash} \PYG{o}{\PYGZhy{}}\PYG{o}{\PYGZhy{}}\PYG{o}{\PYGZhy{}}\PYG{o}{\PYGZhy{}}

\PYG{n}{Rentrez} \PYG{n}{les} \PYG{n}{commandes} \PYG{n}{suivantes} \PYG{n}{dans} \PYG{n}{l}\PYG{l+s+s1}{\PYGZsq{}}\PYG{l+s+s1}{ordre :}
\PYG{l+m+mi}{1}\PYG{p}{)} \PYG{n}{git} \PYG{n}{pull} \PYG{n}{origin} \PYG{l+s+s1}{\PYGZsq{}}\PYG{l+s+s1}{nom de la branche choisie}\PYG{l+s+s1}{\PYGZsq{}}
\PYG{l+m+mi}{2}\PYG{p}{)} \PYG{n}{git} \PYG{n}{switch} \PYG{l+s+s1}{\PYGZsq{}}\PYG{l+s+s1}{nom de la branche choisie}\PYG{l+s+s1}{\PYGZsq{}}
\PYG{l+m+mi}{3}\PYG{p}{)} \PYG{n}{faites} \PYG{n}{vos} \PYG{n}{changement}
\PYG{l+m+mi}{4}\PYG{p}{)} \PYG{n}{git} \PYG{n}{add} \PYG{n}{nom\PYGZus{}du\PYGZus{}fichier}
\PYG{l+m+mi}{5}\PYG{p}{)} \PYG{n}{git} \PYG{n}{commit} \PYG{o}{\PYGZhy{}}\PYG{n}{m} \PYG{l+s+s2}{\PYGZdq{}}\PYG{l+s+s2}{message qui sera afficher}\PYG{l+s+s2}{\PYGZsq{}}
\PYG{l+m+mi}{6}\PYG{p}{)} \PYG{n}{git} \PYG{n}{push} \PYG{n}{origin} \PYG{l+s+s1}{\PYGZsq{}}\PYG{l+s+s1}{nom de la branche choisie}\PYG{l+s+s1}{\PYGZsq{}}
\end{sphinxVerbatim}

\sphinxAtStartPar
Sans ticket

\begin{sphinxVerbatim}[commandchars=\\\{\}]
\PYG{l+m+mi}{1}\PYG{p}{)} \PYG{n}{Creer} \PYG{n}{une} \PYG{n}{branche} \PYG{n}{avec} \PYG{p}{:} \PYG{n}{git} \PYG{n}{branch} \PYG{l+s+s1}{\PYGZsq{}}\PYG{l+s+s1}{nom de la branche}\PYG{l+s+s1}{\PYGZsq{}}
\PYG{l+m+mi}{2}\PYG{p}{)} \PYG{n}{Selectionner} \PYG{n}{la} \PYG{n}{branche} \PYG{n}{avec} \PYG{p}{:} \PYG{n}{git} \PYG{n}{checkout} \PYG{l+s+s1}{\PYGZsq{}}\PYG{l+s+s1}{nom de la branche}\PYG{l+s+s1}{\PYGZsq{}}
\PYG{l+m+mi}{3}\PYG{p}{)} \PYG{n}{Faites} \PYG{n}{vos} \PYG{n}{changement}
\PYG{l+m+mi}{4}\PYG{p}{)} \PYG{n}{git} \PYG{n}{add} \PYG{n}{nom\PYGZus{}du\PYGZus{}ficier}
\PYG{l+m+mi}{5}\PYG{p}{)} \PYG{n}{git} \PYG{n}{commit} \PYG{o}{\PYGZhy{}}\PYG{n}{m} \PYG{l+s+s2}{\PYGZdq{}}\PYG{l+s+s2}{message qui sera afficher}\PYG{l+s+s2}{\PYGZdq{}}
\PYG{l+m+mi}{6}\PYG{p}{)} \PYG{n}{git} \PYG{n}{push} \PYG{n}{origin} \PYG{l+s+s1}{\PYGZsq{}}\PYG{l+s+s1}{nom de la branche}\PYG{l+s+s1}{\PYGZsq{}}

\PYG{o}{\PYGZhy{}}\PYG{o}{\PYGZhy{}}\PYG{o}{\PYGZhy{}}\PYG{o}{\PYGZhy{}} \PYG{n}{Passez} \PYG{n}{sur} \PYG{n}{GITlab} \PYG{o}{\PYGZhy{}}\PYG{o}{\PYGZhy{}}\PYG{o}{\PYGZhy{}}\PYG{o}{\PYGZhy{}}

\PYG{l+m+mi}{1}\PYG{p}{)} \PYG{n}{allez} \PYG{n}{sur} \PYG{n}{votre} \PYG{n}{projet}
\PYG{l+m+mi}{2}\PYG{p}{)} \PYG{n}{tout} \PYG{n}{en} \PYG{n}{haut} \PYG{n}{de} \PYG{n}{la} \PYG{n}{page} \PYG{n}{on} \PYG{n}{va} \PYG{n}{vous} \PYG{n}{proposer} \PYG{n}{de} \PYG{n}{faire} \PYG{n}{un} \PYG{n}{merge} \PYG{n}{request}\PYG{p}{,} \PYG{n}{acceptez}
\PYG{l+m+mi}{3}\PYG{p}{)} \PYG{n}{rentrez} \PYG{n}{les} \PYG{n}{parametres} \PYG{n}{du} \PYG{n}{merge} \PYG{n}{request} \PYG{n}{et} \PYG{n}{c}\PYG{l+s+s1}{\PYGZsq{}}\PYG{l+s+s1}{est bon !}
\end{sphinxVerbatim}

\sphinxstepscope


\chapter{Definition\_des\_fonctions}
\label{\detokenize{Definition_des_fonctions:definition-des-fonctions}}\label{\detokenize{Definition_des_fonctions::doc}}

\section{Boucle Automatisation}
\label{\detokenize{Definition_des_fonctions:module-boucle_automatisation}}\label{\detokenize{Definition_des_fonctions:boucle-automatisation}}\index{module@\spxentry{module}!boucle\_automatisation@\spxentry{boucle\_automatisation}}\index{boucle\_automatisation@\spxentry{boucle\_automatisation}!module@\spxentry{module}}
\sphinxAtStartPar
Dans le fichier « boucle\_automatisation.py, vous trouverez : les fonctions pour ajouter un nom, 
un prénom, la date d’arrivée du nouvel employé, son poste ainsi que le nom de la personne qui l’ajoute.
Le tout est ensuite implémenté dans un dataframe pandas puis transformé en fichier.csv.
\index{ajout\_auto() (dans le module boucle\_automatisation)@\spxentry{ajout\_auto()}\spxextra{dans le module boucle\_automatisation}}

\begin{fulllineitems}
\phantomsection\label{\detokenize{Definition_des_fonctions:boucle_automatisation.ajout_auto}}
\pysigstartsignatures
\pysiglinewithargsret{\sphinxcode{\sphinxupquote{boucle\_automatisation.}}\sphinxbfcode{\sphinxupquote{ajout\_auto}}}{\emph{\DUrole{n}{data}}}{}
\pysigstopsignatures
\sphinxAtStartPar
Ce code définit une fonction appelée ajout\_auto qui prend en paramètre data et renvoie un résultat res. Elle ajoute
une nouvelle ligne au dataframe data à l’index len(data), qui correspond à la dernière ligne du dataframe. La nouvelle
ligne est une liste contenant les valeurs nom, prenom, date, poste, prescri et la date et l’heure courantes.

\sphinxAtStartPar
Pour utiliser cette fonction, vous pouvez l’appeler et lui passer un dataframe en tant que paramètre data. La fonction 
ajoutera alors une nouvelle ligne au dataframe avec les valeurs spécifiées et renverra le résultat.

\end{fulllineitems}

\index{position() (dans le module boucle\_automatisation)@\spxentry{position()}\spxextra{dans le module boucle\_automatisation}}

\begin{fulllineitems}
\phantomsection\label{\detokenize{Definition_des_fonctions:boucle_automatisation.position}}
\pysigstartsignatures
\pysiglinewithargsret{\sphinxcode{\sphinxupquote{boucle\_automatisation.}}\sphinxbfcode{\sphinxupquote{position}}}{\emph{\DUrole{n}{choix\_metier}}}{}
\pysigstopsignatures
\sphinxAtStartPar
Cette fonction définit une fonction appelée position qui prend en paramètre choix\_metier et renvoie la position choisie 
par l’utilisateur. Il utilise la bibliothèque inquirer pour demander à l’utilisateur de sélectionner une position dans 
la liste de choix passée en paramètre de la fonction List. La position sélectionnée par l’utilisateur est renvoyée 
sous forme de chaîne de caractères grâce à la clé position du dictionnaire answers.

\end{fulllineitems}

\index{time() (dans le module boucle\_automatisation)@\spxentry{time()}\spxextra{dans le module boucle\_automatisation}}

\begin{fulllineitems}
\phantomsection\label{\detokenize{Definition_des_fonctions:boucle_automatisation.time}}
\pysigstartsignatures
\pysiglinewithargsret{\sphinxcode{\sphinxupquote{boucle\_automatisation.}}\sphinxbfcode{\sphinxupquote{time}}}{}{}
\pysigstopsignatures
\sphinxAtStartPar
Cette fonction permet de d’indiquer la date au format jj/MM/AAAA

\end{fulllineitems}



\section{Liste Metier}
\label{\detokenize{Definition_des_fonctions:module-liste_metier}}\label{\detokenize{Definition_des_fonctions:liste-metier}}\index{module@\spxentry{module}!liste\_metier@\spxentry{liste\_metier}}\index{liste\_metier@\spxentry{liste\_metier}!module@\spxentry{module}}
\sphinxAtStartPar
Dans le fichier « liste\_metiers.py, vous trouverez la liste des postes au sein dl’entreprise.
\index{metiers() (dans le module liste\_metier)@\spxentry{metiers()}\spxextra{dans le module liste\_metier}}

\begin{fulllineitems}
\phantomsection\label{\detokenize{Definition_des_fonctions:liste_metier.metiers}}
\pysigstartsignatures
\pysiglinewithargsret{\sphinxcode{\sphinxupquote{liste\_metier.}}\sphinxbfcode{\sphinxupquote{metiers}}}{}{}
\pysigstopsignatures
\sphinxAtStartPar
Cette fonction contient une liste appellée metier, elle retroune le contenu de celle\sphinxhyphen{}ci.

\end{fulllineitems}



\chapter{Documentation de la « liste des employés »}
\label{\detokenize{index:documentation-de-la-liste-des-employes}}
\sphinxAtStartPar
Bienvenu dans le guide d’utilisation de la « Liste des Employés ».
Ce logiciel vise à implémenter la liste des employés de votre entreprise.

\sphinxAtStartPar
Je vous conseille de parcourir la documentation dans l’ordre des rubriques.
Une fois que vous aurez réccupéreé le projet (cf : comment réccupérer) ; vous trouverez à l’intérieur :

\sphinxAtStartPar
\_un fichier « doc », contenant les fihier de documentation suivants
\_un fichier « code et outils », contenant le code à faire jouer.


\renewcommand{\indexname}{Index des modules Python}
\begin{sphinxtheindex}
\let\bigletter\sphinxstyleindexlettergroup
\bigletter{b}
\item\relax\sphinxstyleindexentry{boucle\_automatisation}\sphinxstyleindexpageref{Definition_des_fonctions:\detokenize{module-boucle_automatisation}}
\indexspace
\bigletter{l}
\item\relax\sphinxstyleindexentry{liste\_metier}\sphinxstyleindexpageref{Definition_des_fonctions:\detokenize{module-liste_metier}}
\end{sphinxtheindex}

\renewcommand{\indexname}{Index}
\printindex
\end{document}